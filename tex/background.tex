A typical brewing day involves monitoring and keeping a hot liquor tank (HLT) at a desired temperature, then the
brewer has to send the hot water from the HLT to the empty mash tun for pre-heating. After a set amount of time, the water from the mash tun
will be sent back to the HLT to make sure the mash tun is at the desired temperature. While in the process of
pre-heating, the HLT has to be monitored and will be continuously heated up to the desired temperature.

The next step is to start the mash by mixing in the grain with the hot water inside the mash tun. The water inside the mash tun must be kept at a specific temperature for a set amount of time. If the water drops below the desired temperature, a pump is turned on and the water
is sent through a pump to a coil inside the HLT to heat the water.

Once the desired amount of time has passed, the HLT will
be heated up to a higher temperature and then sent to the mash tun to rinse the sugars.
At the same time, beer inside the mash tun will be pumped into the boil kettle.
After all the beer has been sent to the boil kettle, the brewer must boil the beer and set a timer to add hops to the beer
after each cycle. Once the boil is finished, the beer inside the boil kettle will be sent through a chiller to the fermenter.

The process of brewing without automation requires constant monitoring and heating of the temperature of the liquids inside the brewing kettles. Without constant supervision during the brewing process, it would be easy for a beginner to make mistakes and ruin the batch. The main idea of the our project is to simplify the brewing process by monitoring and controlling the temperature of the liquids inside of the different kettles. While also controlling where the liquids will go during all steps of the brewing process.
